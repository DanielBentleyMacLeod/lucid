\documentclass{article}

\usepackage{fancyhdr}
\usepackage{extramarks}
\usepackage{amsfonts}
\usepackage{syntax}
\usepackage{stmaryrd}
\usepackage{mathpartir}

%
% Basic Document Settings
%

\topmargin=-0.45in
\evensidemargin=0in
\oddsidemargin=0in
\textwidth=6.5in
\textheight=9.0in
\headsep=0.25in

\linespread{1.1}

\pagestyle{fancy}
\chead{Tuple flattening}
\lfoot{\lastxmark}
\cfoot{\thepage}

\renewcommand\headrulewidth{0.4pt}
\renewcommand\footrulewidth{0.4pt}

\setlength\parindent{30pt}

\newcommand{\Z}{\mathbb{Z}}
\newcommand{\Zt}{$\Z$}

% Create a relational rule
% [#1] - Additional mathpartir arguments
% {#2} - Name of the rule
% {#3} - Premises for the rule
% {#4} - Conclusions for the rule
\newcommand{\relationRule}[4][]{\inferrule*[lab={\sc #2},#1]{#3}{#4}}

\newcommand{\rel}[1]{\ensuremath{\llbracket {#1} \rrbracket}}
\newcommand{\ttt}{\texttt}
\newcommand{\transform}{\rightsquigarrow}
\newcommand{\proj}{\pi}
\newcommand{\ttuple}{(\tau_1, \ldots, \tau_n)}
\newcommand{\etuple}{(e_1,\ldots,e_n)}
\newcommand{\bool}{\mathrm{bool}}
\newcommand{\integer}{\mathrm{int}}
\newcommand{\option}{\mathrm{option}}
\newcommand{\uoption}[1]{(\bool,#1)}
\newcommand{\opt}[1]{\texttt{opt-#1}}
\newcommand{\varOf}[1]{\texttt{varOf}(#1)}



\begin{document}

We track three contexts:
\begin{enumerate}
\item Local variables L
\item Global variables G
\item Size variables S
\end{enumerate}

Local variables follow ``normal'' typing rules. Global variables use a weakly-ordered type system: that is, in any given handler, they may be used at most once, and must be used in the order they are declared. Size variables are used for our polymorphic integer types; they represent a bitwidth, and can be created either by polymorphic function definitions or by the \texttt{size} declaration.

We treat L as a map, since it's just mimicking standard variable bindings.

We treat S as a set, because we don't actually care what particular values the size variables hold, just that they are bound.

We treat G as a list/stack, since we want to ensure that its variables are used in order (and at most once).

\section*{Values}

\begin{mathpar}
	\relationRule{int-const}{
					v,n \in \Z
	}{
          L,G,S \vdash v\textless n \textgreater \colon \integer\textless n \textgreater
	}
\end{mathpar}

\begin{mathpar}
	\relationRule{int-var}{
						v \in \Z, x \in S
	}{
          L,G,S \vdash v\textless x \textgreater \colon \integer\textless x \textgreater
	}
\end{mathpar}

\section*{Declarations}
Declarations don't evaluate to values, so they don't typecheck in the same way. Instead, each one optionally checks the type of its body (in the assumptions space) and transforms the L,G,S contexts for future declarations.

\begin{mathpar}
	\relationRule{size}{
	\
	}{
          L,G,S \vdash \texttt{size } x = _ \transform L,G,(S \cup \{x\}
	}
\end{mathpar}

\begin{mathpar}
	\relationRule{global}{
	        L,G,S \vdash e : \tau
	}{
          L,G,S \vdash \tau\ x = e \transform L,(G@[x \mapsto \tau ]),S
	}
\end{mathpar}

\begin{mathpar}
	\relationRule{printi}{
	        L,G,S \vdash e : \integer\textless\sigma\textgreater
	}{
          L,G,S \vdash \texttt{print_int}\ e \transform L,G,S
	}
\end{mathpar}

\begin{mathpar}
	\relationRule{func-def}{
					\tau_x = \ll x_1, \dots x_m \gg \rightarrow (y_1 \colon \tau_1) \rightarrow \dots \rightarrow (y_n \colon \tau_n) \rightarrow \tau_{ret} \\
	        L' = L[y_1 := \tau_1]...[y_n := \tau_n]\\
					S' = S \cup \{x_1, \dots, x_m\}\\
					L',G,S' \vdash s : \tau_{ret}
	}{
          L,G,S \vdash \texttt{fun}\ \ll x_1, \dots, x_m \gg \tau_{ret}\ x(\tau_1\ y_1, \dots, \tau_n\ y_n)\ \{ s \} \transform L'[x := \tau_x],G,S'
	}
\end{mathpar}

\begin{mathpar}
	\relationRule{event-def}{
					\tau_x = \texttt{TEvent}(\tau_1, \tau_n)
	}{
          L,G,S \vdash \texttt{event}\ x(\tau_1\ y_1, \dots, \tau_n\ y_n)\ \transform L[x := \tau_x],G,S
	}
\end{mathpar}

\begin{mathpar}
	\relationRule{handler}{
					L[x] = \texttt{TEvent}(\tau_1, \tau_n)\\
	}{
          L,G,S \vdash \texttt{handle}\ x(\tau_1\ y_1, \dots, \tau_n\ y_n)\ \{ s \} \transform L,G,S
	}
\end{mathpar}

\end{document}
%%% Local Variables:
%%% mode: latex
%%% TeX-master: t
%%% End:
